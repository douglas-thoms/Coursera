\documentclass[]{article}
\usepackage{lmodern}
\usepackage{amssymb,amsmath}
\usepackage{ifxetex,ifluatex}
\usepackage{fixltx2e} % provides \textsubscript
\ifnum 0\ifxetex 1\fi\ifluatex 1\fi=0 % if pdftex
  \usepackage[T1]{fontenc}
  \usepackage[utf8]{inputenc}
\else % if luatex or xelatex
  \ifxetex
    \usepackage{mathspec}
  \else
    \usepackage{fontspec}
  \fi
  \defaultfontfeatures{Ligatures=TeX,Scale=MatchLowercase}
\fi
% use upquote if available, for straight quotes in verbatim environments
\IfFileExists{upquote.sty}{\usepackage{upquote}}{}
% use microtype if available
\IfFileExists{microtype.sty}{%
\usepackage{microtype}
\UseMicrotypeSet[protrusion]{basicmath} % disable protrusion for tt fonts
}{}
\usepackage[margin=1in]{geometry}
\usepackage{hyperref}
\hypersetup{unicode=true,
            pdftitle={Assignment instructions},
            pdfauthor={Douglas Thoms},
            pdfborder={0 0 0},
            breaklinks=true}
\urlstyle{same}  % don't use monospace font for urls
\usepackage{graphicx,grffile}
\makeatletter
\def\maxwidth{\ifdim\Gin@nat@width>\linewidth\linewidth\else\Gin@nat@width\fi}
\def\maxheight{\ifdim\Gin@nat@height>\textheight\textheight\else\Gin@nat@height\fi}
\makeatother
% Scale images if necessary, so that they will not overflow the page
% margins by default, and it is still possible to overwrite the defaults
% using explicit options in \includegraphics[width, height, ...]{}
\setkeys{Gin}{width=\maxwidth,height=\maxheight,keepaspectratio}
\IfFileExists{parskip.sty}{%
\usepackage{parskip}
}{% else
\setlength{\parindent}{0pt}
\setlength{\parskip}{6pt plus 2pt minus 1pt}
}
\setlength{\emergencystretch}{3em}  % prevent overfull lines
\providecommand{\tightlist}{%
  \setlength{\itemsep}{0pt}\setlength{\parskip}{0pt}}
\setcounter{secnumdepth}{0}
% Redefines (sub)paragraphs to behave more like sections
\ifx\paragraph\undefined\else
\let\oldparagraph\paragraph
\renewcommand{\paragraph}[1]{\oldparagraph{#1}\mbox{}}
\fi
\ifx\subparagraph\undefined\else
\let\oldsubparagraph\subparagraph
\renewcommand{\subparagraph}[1]{\oldsubparagraph{#1}\mbox{}}
\fi

%%% Use protect on footnotes to avoid problems with footnotes in titles
\let\rmarkdownfootnote\footnote%
\def\footnote{\protect\rmarkdownfootnote}

%%% Change title format to be more compact
\usepackage{titling}

% Create subtitle command for use in maketitle
\newcommand{\subtitle}[1]{
  \posttitle{
    \begin{center}\large#1\end{center}
    }
}

\setlength{\droptitle}{-2em}

  \title{Assignment instructions}
    \pretitle{\vspace{\droptitle}\centering\huge}
  \posttitle{\par}
    \author{Douglas Thoms}
    \preauthor{\centering\large\emph}
  \postauthor{\par}
      \predate{\centering\large\emph}
  \postdate{\par}
    \date{June 8, 2019}


\begin{document}
\maketitle

\hypertarget{instructions}{%
\subsection{Instructions}\label{instructions}}

Part 1: Simulation Exercise Instructionsless In this project you will
investigate the exponential distribution in R and compare it with the
Central Limit Theorem. The exponential distribution can be simulated in
R with rexp(n, lambda) where lambda is the rate parameter. The mean of
exponential distribution is 1/lambda and the standard deviation is also
1/lambda. Set lambda = 0.2 for all of the simulations. You will
investigate the distribution of averages of 40 exponentials. Note that
you will need to do a thousand simulations.

Illustrate via simulation and associated explanatory text the properties
of the distribution of the mean of 40 exponentials. You should

Show the sample mean and compare it to the theoretical mean of the
distribution. Show how variable the sample is (via variance) and compare
it to the theoretical variance of the distribution. Show that the
distribution is approximately normal. In point 3, focus on the
difference between the distribution of a large collection of random
exponentials and the distribution of a large collection of averages of
40 exponentials.

As a motivating example, compare the distribution of 1000 random
uniforms

 and the distribution of 1000 averages of 40 random uniforms

 This distribution looks far more Gaussian than the original uniform
distribution!

This exercise is asking you to use your knowledge of the theory given in
class to relate the two distributions.

Confused? Try re-watching video lecture 07 for a starter on how to
complete this project.

Sample Project Report Structure

Of course, there are multiple ways one could structure a report to
address the requirements above. However, the more clearly you pose and
answer each question, the easier it will be for reviewers to clearly
identify and evaluate your work.

A sample set of headings that could be used to guide the creation of
your report might be:

Title (give an appropriate title) and Author Name Overview: In a few
(2-3) sentences explain what is going to be reported on. Simulations:
Include English explanations of the simulations you ran, with the
accompanying R code. Your explanations should make clear what the R code
accomplishes. Sample Mean versus Theoretical Mean: Include figures with
titles. In the figures, highlight the means you are comparing. Include
text that explains the figures and what is shown on them, and provides
appropriate numbers. Sample Variance versus Theoretical Variance:
Include figures (output from R) with titles. Highlight the variances you
are comparing. Include text that explains your understanding of the
differences of the variances. Distribution: Via figures and text,
explain how one can tell the distribution is approximately normal. Part
2: Basic Inferential Data Analysis Instructionsless Now in the second
portion of the project, we're going to analyze the ToothGrowth data in
the R datasets package.

Load the ToothGrowth data and perform some basic exploratory data
analyses Provide a basic summary of the data. Use confidence intervals
and/or hypothesis tests to compare tooth growth by supp and dose. (Only
use the techniques from class, even if there's other approaches worth
considering) State your conclusions and the assumptions needed for your
conclusions.

\hypertarget{summary}{%
\subsection{Summary}\label{summary}}

\begin{enumerate}
\def\labelenumi{\arabic{enumi}.}
\tightlist
\item
  First set up simulation The exponential distribution can be simulated
  in R with rexp(n, lambda) where lambda is the rate parameter. The mean
  of exponential distribution is 1/lambda and the standard deviation is
  also 1/lambda. Set lambda = 0.2 for all of the simulations. You will
  investigate the distribution of averages of 40 exponentials. Note that
  you will need to do a thousand simulations.
\end{enumerate}


\end{document}
